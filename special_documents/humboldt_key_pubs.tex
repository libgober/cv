\documentclass[12pt]{article}
\usepackage[utf8]{inputenc}
\usepackage[margin=1in]{geometry} % Sets 1-inch margins
\usepackage{enumitem}
\usepackage{tgpagella} % This package is used for a Garamond-like font
\title{\large\bfseries List of the selected key publications of Brian Libgober}
\date{}


\usepackage[backend=biber,isbn=false,eprint=false,url=false,maxcitenames=50,style=chicago-authordate]{biblatex} %bib files
\addbibresource{LibgoberPublications.bib}

\newcommand{\makeauthorbold}[1]{%
  \DeclareNameFormat{author}{%
    \ifthenelse{\value{listcount}=1}
    {%
      {\expandafter\ifstrequal\expandafter{\namepartfamily}{#1}{\mkbibbold{\namepartfamily\addcomma\addspace \namepartgiveni}}{\namepartfamily\addcomma\addspace \namepartgiveni}}
      %
    }{\ifnumless{\value{listcount}}{\value{liststop}}
        {\expandafter\ifstrequal\expandafter{\namepartfamily}{#1}{\mkbibbold{\addcomma\addspace \namepartfamily\addcomma\addspace \namepartgiveni}}{\addcomma\addspace \namepartfamily\addcomma\addspace \namepartgiveni}}
        {\expandafter\ifstrequal\expandafter{\namepartfamily}{#1}{\mkbibbold{\addcomma\addspace \namepartfamily\addcomma\addspace \namepartgiveni\addcomma\isdot}}{\addcomma\addspace \namepartfamily\addcomma\addspace \namepartgiveni\addcomma\isdot}}%
      }
    \ifthenelse{\value{listcount}<\value{liststop}}
    {\addcomma\space}{}
  }
}
\makeauthorbold{Libgober}

\begin{document}

\maketitle
\vspace{-20mm} % Reduces the space between the title and the start of the document

\begin{enumerate}
\item \fullcite{LibgoberCarpenter2024}

\begin{quote}
 One of the two research articles supporting a recent book manuscript proposal. This study uniquely quantifies and qualifies the extensive yet obscured activities of lawyers in financial regulatory advocacy, a sector that operates largely outside the purview of existing US lobbying disclosure laws. I argue through multiple methods that the scale and financial investment in undisclosed US regulatory lobbying likely exceeds the investment in reported, largely legislatively focused lobbying. Leveraging multiple new data streams, including hand-coded archival records, financial reporting forms, and personnel databases, the paper offers insight into the dynamics of political influence. \emph{Perspectives on Politics} is one of the leading journals in political science, particularly for public-facing, agenda-setting, and richly descriptive works like this one. 
\end{quote}


\item \fullcite{Libgober2020b}

\begin{quote}
The second of two key research articles supporting a book proposal that is currently under review. The paper presents evidence about the efficacy of administratively focused lobbying. It leverages data on who meets and comments on regulations and embeds these within a stock market event study. In particular, I find that firms lobbying the Federal Reserve about regulations implementing the 2010 Dodd Frank Wall Street Reform Act received unusually large boosts to their stock price after regulations were released. Firms that met with regulators early in the regulation development process received market valuations that were billions of dollars higher than the valuations of firms that did not obtain such privileged access.  The paper is now recognized for providing some of the strongest direct evidence of lobbying's effectiveness in political science and has started to appear in administrative law and political economy syllabi across various academic institutions. 
\end{quote}

\item \fullcite{Libgober2020}

\begin{quote}
My most prominent formal theoretic work, which also appears in a top three political science journal. It proposes a model for analyzing participatory trends in notice-and-comment rulemaking. In particular, this framework supposes that regulators are Bayesian and interest groups' decisions to participate in regulatory processes are strategic, in the sense that interest groups consider the likely inferences that the regulator will make about their decision not to participate. The paper uses this model to reevaluate existing evidence about participation in terms of regulatory capture and bureaucratic autonomy. This work not only redefines existing theoretical landscapes but also encourages the development of new methodologies for assessing regulatory influence and public policy outcomes.
\end{quote}

\item \fullcite{Libgober2019}

\begin{quote}
My most prominent publication in public administration and public policy, placed by many estimates in the leading journal of public administration JPART. The paper examines the role of regulations on regulation making processes, in particular, it follows recent trends in behavioral public administration in seeking to evaluate the benefits and cost of cost benefit analysis. In contrast with the hypothesis of prominent regulatory analyst Cass Sunstein, we find no uniform benefits of cost benefit analysis in terms of reducing cognitive bias. We explain our mixed findings by appeal to cognitive psychology, and argue that cost-benefit analysis is only likely to ameliorate biases rooted in affective (i.e. emotional) thinking. We discuss implications for the design of regulatory process and future empirical work. 
\end{quote}

\item \fullcite{ChenGober2023}

\begin{quote}
My first article on issues of access to courts and civil justice, which is the basis of a nearly half million dollar awards from the US National Science Foundation that will support numerous experiments and research papers going forward (e.g. a recently accepted paper at the journal of causal inference, a chapter in an edited volume for “new voices” in this interdisciplinary subfield, etc). This paper presents evidence of extensive racial discrimination in legal markets through audit experiments. I develop in this paper, a theory of racial service rationing to explain variation in the degree of discrimination, observed in lawyers, decision to respond to request for representation from apparently white or black clients. The paper has been discussed in numerous media outlets and has been reviewed favorably on JOTWELL, a well-known blog reviewing notable, high-quality legal scholarship. 
\end{quote}

\end{enumerate}
\end{document}
