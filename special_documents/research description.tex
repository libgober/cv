"White Shoes, Hidden Hands" examines the vast, hidden world of administrative lobbying in the United States, focusing on financial regulation as a key case. Working from the emerging American Political Economy framework, the book argues that the US delegates vast policymaking authorities to executive agencies while encouraging a largely unregulated, unaccountable, and nontransparent approach to stakeholder engagement and economic governance. A key contrast that the book must address is how the US approach differs from other national models. Germany's coordinated approach to governing markets is a classic contrast with the more laissez faire approach of the US. Moreover, Germany has a distinctive and relatively intentional approach to the design of stakeholder engagements in regulatory design. For these reasons, I am hoping to develop a book chapter describing the German approach to the same or similar problems as I address primarily in the United States. 



My research in Germany will develop my academic profile in at least four key ways.

1. It will deepen the quality of my book manuscript, "White Shoes, Hidden Hands," by allowing me to test its hypotheses, conclusions, and perspective in a different national context. This experience is likely to lead to a more thoughtful and nuanced manuscript than I will be able to arrive at by examining only one national case.

2. By refining and engaging with questions about the comparative design of institutions and financial regulation, I will substantially broaden the audience of this work by making it far more relevant to scholars working in comparative political economy, comparative politics, and comparative public policy. 

3. The opportunity to study the financial regulatory policy choices of Germany will help me to develop uncommon and potentially valuable perspective on the policy issues inherent in financial regulation as well as the space of possible policy solutions. In so doing, it will enhance my knowledge and expertise in this important policy area at the core of my work.

4. The research experience will help foster connections between myself and German scholars specifically and European scholars more generally, laying a foundation for future collaborations and also increasing the size of my audience.

