\documentclass[12pt]{article}
\usepackage[utf8]{inputenc}
\usepackage[margin=1in]{geometry} % Sets 1-inch margins
\usepackage{enumitem}
\usepackage{tgpagella} % This package is used for a Garamond-like font
\title{\large\bfseries List of the selected key publications of Brian Libgober}
\date{}


\usepackage[backend=biber,isbn=false,eprint=false,url=false,maxcitenames=50,style=chicago-authordate]{biblatex} %bib files
\addbibresource{LibgoberPublications.bib}

\newcommand{\makeauthorbold}[1]{%
  \DeclareNameFormat{author}{%
    \ifthenelse{\value{listcount}=1}
    {%
      {\expandafter\ifstrequal\expandafter{\namepartfamily}{#1}{\mkbibbold{\namepartfamily\addcomma\addspace \namepartgiveni}}{\namepartfamily\addcomma\addspace \namepartgiveni}}
      %
    }{\ifnumless{\value{listcount}}{\value{liststop}}
        {\expandafter\ifstrequal\expandafter{\namepartfamily}{#1}{\mkbibbold{\addcomma\addspace \namepartfamily\addcomma\addspace \namepartgiveni}}{\addcomma\addspace \namepartfamily\addcomma\addspace \namepartgiveni}}
        {\expandafter\ifstrequal\expandafter{\namepartfamily}{#1}{\mkbibbold{\addcomma\addspace \namepartfamily\addcomma\addspace \namepartgiveni\addcomma\isdot}}{\addcomma\addspace \namepartfamily\addcomma\addspace \namepartgiveni\addcomma\isdot}}%
      }
    \ifthenelse{\value{listcount}<\value{liststop}}
    {\addcomma\space}{}
  }
}
\makeauthorbold{Libgober}

\begin{document}

\maketitle
\vspace{-20mm} % Reduces the space between the title and the start of the document

\begin{enumerate}
\item \fullcite{LibgoberCarpenter2024}

\begin{quote}
 This study, for the first time, quantifies and qualifies the extensive yet obscured activities of lawyers in financial regulatory advocacy, a sector that operates largely outside the purview of existing US lobbying disclosure laws. Libgober and Carpenter argue through multiple methods that the scale and financial investment in undisclosed US regulatory lobbying likely exceeds the investment in reported, largely legislatively focused lobbying. Leveraging multiple new data streams, including hand-coded archival records, financial reporting forms, and personnel databases, the paper offers insight into the dynamics of political influence. \emph{Perspectives on Politics} is one of the leading journals in political science, particularly for public-facing, agenda-setting, and richly descriptive works like this one. 
\end{quote}


\item \fullcite{Libgober2020b}

\begin{quote}
This highly original article presents remarkable evidence about the efficacy of administratively focused lobbying. It leverages unique, difficult-to-collect data on who meets and comments on regulations and embeds these within a stock market event study. In particular, Libgober finds that firms lobbying the Federal Reserve about regulations implementing the 2010 Dodd Frank Wall Street Reform Act received unusually large boosts to their stock price after regulations were released. Firms that met with regulators early in the regulation development process received market valuations that were billions of dollars higher than the valuations of firms that did not obtain such privileged access.  The paper has been recognized for providing some of the strongest direct evidence of the effectiveness of lobbying in the contemporary social science literature. The article has appeared in recent administrative law and political economy syllabi across various academic institutions. 
\end{quote}

\item \fullcite{Libgober2020}

\begin{quote}
The first formal model of the notice-and-comment rulemaking process, which also appears in a top three political science journal. In Libgober's model, regulators are Bayesian and interest groups' decisions to participate in regulatory processes are strategic, in the sense that interest groups consider the likely inferences that the regulator will make about their decision not to participate. Libgober draws some novel and important implications from this model, perhaps most interestingly that regulators will strategically tilt their proposals to induce more revelation of information from firms whose viewpoints are least predictable ex ante. The regulator's ability to elicit information depends on the stickiness of the proposal, however, and so more effective elicitation of information produces greater policy bias. Besides drawing out implications of the model for regulatory policy choice, the article also uses the model to reevaluate existing evidence about participation in terms of regulatory capture and bureaucratic autonomy. Generally, Libgober finds that highly skewed participation patterns are poor indicators of which side between say industry and consumers that the regulator actually favors through policy selection. Most intriguingly, even if participation is associated with directional changes in policy favoring participants, that is still potentially consistent with policy highly favorable to those who do not participate. This work not only redefines existing theoretical landscapes but also encourages the development of new methodologies for assessing regulatory influence and public policy outcomes.
\end{quote}

\item \fullcite{ChenGober2023}

\begin{quote}
Libgober's most prominent publication in public administration and public policy, placed by many estimates in the leading journal of public administration JPART. The paper examines the role of regulations on regulation making processes, in particular, it follows recent trends in behavioral public administration in seeking to evaluate the benefits and cost of cost benefit analysis. In contrast with the hypothesis of prominent regulatory analyst Cass Sunstein, we find no uniform benefits of cost benefit analysis in terms of reducing cognitive bias. Libgober and Chen explain our mixed findings by appeal to cognitive psychology, and argue that cost-benefit analysis is only likely to ameliorate biases rooted in affective (i.e. emotional) thinking. They discuss implications for the design of regulatory process and future empirical work. A terrific, recent entry on the fast growing literature on behavioral public administration.
\end{quote}

\item \fullcite{Libgober2019}

\begin{quote}
Libgober's most prominent paper on  issues of access to courts and civil justice. The paper is also the basis of a nearly half million dollar awards from the US National Science Foundation. Libgober presents evidence of extensive racial discrimination in legal markets through audit experiments. He develops in this paper a theory of racial service rationing to explain variation in the degree of discrimination, observed in lawyers, decision to respond to request for representation from apparently white or black clients. The paper has been discussed in numerous media outlets and has been reviewed favorably on JOTWELL, a well-known blog reviewing notable, high-quality legal scholarship. 
\end{quote}

\end{enumerate}
\end{document}
